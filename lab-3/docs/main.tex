\documentclass[a4paper,12pt]{article}

\usepackage[hidelinks]{hyperref}
\usepackage{amsmath}
\usepackage{mathtools}
\usepackage{shorttoc}
\usepackage{cmap}
\usepackage[T2A]{fontenc}
\usepackage[utf8]{inputenc}
\usepackage[english, russian]{babel}
\usepackage{xcolor}
\usepackage{graphicx}
\usepackage{float}
\graphicspath{{./images/}}

\definecolor{linkcolor}{HTML}{000000}
\definecolor{urlcolor}{HTML}{0085FF}
\hypersetup{pdfstartview=FitH,
  linkcolor=linkcolor,
  urlcolor=urlcolor,
colorlinks=true}

\DeclarePairedDelimiter{\floor}{\lfloor}{\rfloor}

\renewcommand*\contentsname{Содержание}

\newcommand{\plot}[3]{
  \begin{figure}[H]
    \begin{center}
      \includegraphics[scale=0.6]{#1}
      \caption{#2}
      \label{#3}
    \end{center}
  \end{figure}
}

\begin{document}
\begin{titlepage}
  \begin{center}
    {\large Санкт-Петербургский политехнический университет\\Петра Великого\\}
  \end{center}

  \begin{center}
    {\large Физико-механический иститут}
  \end{center}


  \begin{center}
    {\large Кафедра «Прикладная математика»}
  \end{center}

  \vspace{8em}

  \begin{center}
    {\bfseries Отчёт по лабораторной работе №3 \\по дисциплине «Анализ данных с интервальной неопределённостью»}
  \end{center}

  \vspace{5em}

  \begin{flushleft}
    \hspace{16em}Выполнил студент:\\\hspace{16em}Мишутин Дмитрий Валерьевич\\\hspace{16em}группа: 5040102/20201

    \vspace{2em}

    \hspace{16em}Проверил:\\\hspace{16em}к.ф.-м.н., доцент\\\hspace{16em}Баженов Александр Николаевич

  \end{flushleft}


  \vspace{6em}


  \begin{center}
    Санкт-Петербург\\2024 г.
  \end{center}

\end{titlepage}

\newpage

\tableofcontents
\listoffigures
\newpage

\section{Постановка задачи}
\quad Провести анализ остатков интервальных измерений.

\section{Теория}
\subsection{Классификация измерений}
\quad Измерения можно классифицировать следующим образом.
Измерения, добавление которых к выборке не приводит к модификации модели, называются \textsl{внутренними}.
Те, которые изменяют модель, называются \textsl{внешними}.
Измерения, которые определяют какую-либо границу информационного множества, называются \textsl{граничными}.
\textsl{Выбросами} называются те измерения, которые делают информационное множество пустым.
Граничные измерения - подмножество внутренних, выбросы - внешних.

Для удобства анализа взаимоотношения информационных множеств работу с ними заменяют
на анализ взаимоотношения интересующего интервального измерения и интервального прогнозируемого
значения модели (коридора совместных значений).

\subsection{Взаимные отношения интервалов наблюдения и прогнозного интервала модели}
\quad Существует несколько характеристик, определяющих это взаимоотношение.

\textsl{Размахом (плечом)} называется следующее отношение \ref{e:leverage}.
\begin{equation}
  l(x, \textbf{y}) = \frac{\Upsilon(x)}{rad(\textbf{y})}
  \label{e:leverage}
\end{equation}

\textsl{Относительным остатком} называется отношение \ref{e:residual}.
\begin{equation}
  r(x, \textbf{y}) = \frac{mid(\textbf{y}) - mid(\Upsilon(x))}{rad(\textbf{y})}
  \label{e:residual}
\end{equation}
здесь $ x $ - точечное значение, $ \textbf{y} $ - интервальное значение интересующей величины (отклик $ x $),
$ \Upsilon(x) $ - интервальная оценка интересующей величины (значение коридора совместных значений).

Для внутренних наблюдений выполняется неравенство \ref{e:inner}.
\begin{equation}
  |r(x, \textbf{y})| \leq 1 - l(x, \textbf(y))
  \label{e:inner}
\end{equation}
В случае равенства \ref{e:inner} измерение будет граничным.

Выбросы определяются неравенством \ref{e:remainder}
\begin{equation}
  |r(x, \textbf{y})| > 1 + l(x, \textbf{y})
  \label{e:remainder}
\end{equation}

\section{Реализация}
\quad Из языка Python 3.12.2 были использованы следующие модули:
\begin{itemize}
  \item ``numpy'' --- генерация множества чисел;
  \item ``matplotlib.pyplot'' --- построение и отображение графиков;
  \item ``scipy'' --- для выполнения научных и инженерных расчётов;
  \item ``glob'' --- расширение шаблонов пути в стиле Unix.
\end{itemize}

\section{Результаты}
\quad Данные $ S_X $ были взяты из файлов \textsl{data/dataset2/XV\_spN.txt}, \newline
где $ X \in P = \{-0.45, -0.35, -0.25, -0.15, -0.05, 0.05, 0.15, 0.25, 0.35, 0.45 \} $.
Набор $ \delta_i $ получен из файла \textsl{data/dataset2/0.0V\_sp443.txt}.

Рассмотрим первую выборку $ Y_1 $. $ Y_1 $ получена следующим образом.
$ \textbf{y}_i = [\min{S_i}, \max{S_i}]$, $ i \in P $, $ \textbf{y}_i \in Y_1 $.
\plot{./images/SampleX1}{Первая выборка, $ Y_1 $}{p:sampleX1}

Построим точечную линейную регрессию и коридор совместных значений.
\plot{./images/InformSetCorridorX1}{Точечная линейная регрессия и коридор совместных значений для $ Y_1 $}{p:informSetCorridorX1}

Построим выборку остатков $ \mathcal{E}_1 $, $ \varepsilon_i = \textbf{y}_i - (\beta_0 + \beta_1 x_i) $,
$ \varepsilon_i \in \mathcal{E}_1 $.

Выборка $ \mathcal{E}_1 $ и коридор совместных значений для $ \mathcal{E}_1 $ имеют вид.
\plot{./images/InformSetCorridorRemX1}{Точечная линейная регрессия и коридор совместных значений для $ \mathcal{E}_1 $}{p:informSetCorridorRemX1}

Теперь построим диаграмму статусов для выборки $ \mathcal{E}_1 $.
По оси $ x $ лежит значение размаха (см. \ref{e:leverage}), по оси $ y $ значение относительного остатка (см. \ref{e:residual}).
\plot{./images/DiagramStatusX1}{Диаграмма статусов измерений выборки $ \mathcal{E}_1 $}{p:diagramStatusX1}

Для данной выборки $ \mathcal{E}_1 $ и простейшей линейной модели граничными являются
измерения, соответствующие следующим значениям переменной $ x $: $ [-0.45, -0.35, -0.25, -0.05, 0.35] $.
Измерение, соответствующее переменной $ x = 0.45 $, возможно, является внешним или также граничным,
а все остальные измерения внутренние (рис. \ref{p:diagramStatusZoomX1}).
\plot{./images/DiagramStatusZoomX1}{Диаграмма статусов измерений выборки $ \mathcal{E}_1 $ (Приближеие)}{p:diagramStatusZoomX1}

Для наглядности проведём аналогичные измерения для другой выборки $ Y_2 $.
$ Y_2 $ получена следующим образом. $ \textbf{y}_i = [median(S_i) - \varepsilon, median(S_i) + \varepsilon] $,
$ \varepsilon = 25.0 $, $ i \in P $, $ \textbf{y}_i \in Y_2 $.

$ Y_2 $ имеет вид.
\plot{./images/SampleX2}{Вторая выборка, $ Y_2 $}{p:sampleX2}

Построим точечную линейную регрессию и коридор совместных значений для $ Y_2 $.
\plot{./images/InformSetCorridorX2}{Точечная линейная регрессия и коридор совместных значений для $ Y_2 $}{p:informSetCorridorX2}

Выборка остатков $ \mathcal{E}_2 $ и коридор совместных значений для $ \mathcal{E}_2 $ имеют вид.
\plot{./images/InformSetCorridorRemX2}{Точечная линейная регрессия и коридор совместных значений для $ \mathcal{E}_2 $}{p:informSetCorridorRemX2}

Построим диаграмму статусов для $ \mathcal{E}_2 $.
\plot{./images/DiagramStatusX2}{Диаграмма статусов измерений выборки $ \mathcal{E}_2 $}{p:diagramStatusX2}

Для выборки $ \mathcal{E}_2 $ граничными являются измерения, соответствующие значениям
переменной $ x \in [-0.45, -0.15, -0.05, 0.35, 0.45] $.
Остальные являются внутренними.
\plot{./images/DiagramStatusZoomX2}{Диаграмма статусов измерений выборки $ \mathcal{E}_2 $ (Приближение)}{p:diagramStatusZoomX2}


\section{Выводы}
\quad Из полученных результатов можно заметить следующее. Для первой выборки на
диаграмме статутов измерений статусы находятся вблизи точки $ (1, 0) $ (рис.
\ref{p:diagramStatusX1}), в то время как для второй выборки статусы
расположились дальше от точки $ (1, 0) $, имеют меньшие значения для плеча (см.
\ref{e:leverage}) и большие по модулю для относительного остатка (см.
\ref{e:residual}). Это вполне сочетается с тем, как выглядят коридоры совместных
значений для каждой выборки (рис. \ref{p:informSetCorridorRemX1},
\ref{p:informSetCorridorRemX2}). Также стоит отметить, что ни для одной выборки
не было обнаружено выбросов или явных внешних измерений.

\section{Литература}
\begin{itemize}
  \item \href{https://elib.spbstu.ru/dl/2/s20-76.pdf/info}{Баженов А.Н.
    <<Интервальный анализ. Основы теории и учебные примеры: учебное пособие>>};
  \item \href{https://elib.spbstu.ru/dl/5/tr/2021/tr21-169.pdf/info}{Баженов
      А.Н. <<Естественнонаучные и технические применения интервального анализа:
    учебное пособие>>};
  \item \href{https://github.com/AlexanderBazhenov/Students}{Баженов А.Н.
    Репозиторий ``Students'' на GitHub};
\end{itemize}

\section{Приложения}
\quad Исходники лабораторной работы выложены на
\href{https://github.com/MeShootIn/interval-analysis/tree/lab-3}{GitHub}.

\end{document}
