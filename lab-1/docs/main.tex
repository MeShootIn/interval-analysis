\documentclass[a4paper,12pt]{article}

\usepackage[hidelinks]{hyperref}
\usepackage{amsmath}
\usepackage{mathtools}
\usepackage{shorttoc}
\usepackage{cmap}
\usepackage[T2A]{fontenc}
\usepackage[utf8]{inputenc}
\usepackage[english, russian]{babel}
\usepackage{xcolor}
\usepackage{graphicx}
\usepackage{float}
\graphicspath{{./images/}}

\definecolor{linkcolor}{HTML}{000000}
\definecolor{urlcolor}{HTML}{0085FF}
\hypersetup{pdfstartview=FitH,
  linkcolor=linkcolor,
  urlcolor=urlcolor,
colorlinks=true}

\DeclarePairedDelimiter{\floor}{\lfloor}{\rfloor}

\renewcommand*\contentsname{Содержание}

\newcommand{\plot}[3]{
  \begin{figure}[H]
    \begin{center}
      \includegraphics[scale=0.6]{#1}
      \caption{#2}
      \label{#3}
    \end{center}
  \end{figure}
}

\newcommand{\GoBackN}{``Go-Back-N''}
\newcommand{\SelectiveRepeat}{``Selective Repeat''}

\begin{document}
\begin{titlepage}
  \begin{center}
    {\large Санкт-Петербургский политехнический университет\\Петра Великого\\}
  \end{center}

  \begin{center}
    {\large Физико-механический иститут}
  \end{center}


  \begin{center}
    {\large Кафедра «Прикладная математика»}
  \end{center}

  \vspace{8em}

  \begin{center}
    {\bfseries Отчёт по лабораторной работе №3 \\по дисциплине «Анализ данных с интервальной неопределённостью»}
  \end{center}

  \vspace{5em}

  \begin{flushleft}
    \hspace{16em}Выполнил студент:\\\hspace{16em}Мишутин Дмитрий Валерьевич\\\hspace{16em}группа: 5040102/20201

    \vspace{2em}

    \hspace{16em}Проверил:\\\hspace{16em}к.ф.-м.н., доцент\\\hspace{16em}Баженов Александр Николаевич

  \end{flushleft}


  \vspace{6em}


  \begin{center}
    Санкт-Петербург\\2024 г.
  \end{center}

\end{titlepage}

\newpage

\tableofcontents
\listoffigures
\newpage

\section{Постановка задачи}
\quad Имеется две вещественные выборки.
Необходимо на их основе построить две интервальные выборки $ X_{1}, X_{2} $.
Рассматриваются 4 меры совместности интервальных данных: индекс Жаккара,
частота моды, оптимальный корректирующий множитель в методе центра
неопределённости, мультипликативная мера.
Каждую выборку необходимо оценить в отдельности с использованием
каждой меры совместности. После, для каждой меры найти такие значения $ R $,
которые были бы оптимальным для выборки $ X = X_{1} \cup R X_{2} $.

\section{Теория}
\subsection{Индекс Жаккара}
\quad Индекс Жаккара (далее обозначим через $ JC $) определяет степень
совместности двух интервалов $ x, y $.
\begin{equation}
  JC(x, y) = \frac{wid(x \land y)}{wid(x \lor y)}
  \label{e:simplejaccard}
\end{equation}

Здесь $\land, \lor$ представляют собой операции взятия минимума и
максимума по включению в полной арифметике Каухера.
Формула \ref{e:simplejaccard} легко может быть обобщена на случай
интервальной выборки $ X = \{x_i\}_{i=1}^{n} $.

\begin{equation}
  JC(X) = \frac{wid(\land_{i=1,n}x_i)}{wid(\lor_{i=1,n}x_i)}
  \label{e:jaccard}
\end{equation}

\subsection{Частота моды}
\quad Модой интервальной выборки называют совокупность интервалов
пересечения наибольших совместных подвыборок рассматриваемой
выборки. Наибольшая длина совместных подвыборок данной выборки называется
частотой моды. Исследование частоты моды (обозначим далее через
$ max\mu $) имеет смысл только для несовместных выборок.

\subsection{Оптимальный корректирующий множитель в методе центра
неопределённости}
\quad Для обеспечения совместности выборки интервальных измерений
применяется метод ``центра неопределенности''. Если выборка измерений
несовместна, то путем одновременного увеличения величины неопределенности
всех измерений в выражении можно всегда добиться того, чтобы выборка стала
совместной.

\begin{equation}
  \overline{x_{i}^{'}} = \overline{x_{i}} + k * (\overline{x_{i}} - \underline{x_{i}})
\end{equation}
\begin{equation}
  \underline{x_{i}^{'}} = \underline{x_{i}} - k * (\overline{x_{i}} - \underline{x_{i}})
\end{equation}

Оптимальным ($ k_{0} $) называется такое значение $ k $, при котором
непустое пересечение интервальной выборки является точкой.

\subsection{Мультипликативная мера}
\quad Мультипликативная мера $ T $ учитывает степень совместности по
нескольким функционалам качества одновременно, а ее значения находятся
в интервале $ [0, 1] $:

\begin{equation}
  T = \prod_{i=1}^k T_i
  \label{e:Mult}
\end{equation}

В качестве множителей берем перечисленные выше меры, нормированные в
$ [0, 1] $:

\begin{equation}
  T_1 = \frac{1}{2} * (1 + JC)
\end{equation}
\begin{equation}
  T_2 = \frac{max \mu}{n}
\end{equation}
\begin{equation}
  T_3 = \frac{1}{1 + k_{0}}
\end{equation}

\subsection{Нахождение оптимального значения R}
\quad Для поиска оптимальных значений $ R $ разумно найти первое
приближение. Возьмем за такое приближение внешнюю оценку $ R_{out} $.

\begin{equation}
  \underline{R_{out}} = \frac{\min_{i=1,n}\underline{x_{1i}}}{\max_{i=1,n}\overline{x_{2i}}}
\end{equation}
\begin{equation}
  \overline{R_{out}} = \frac{\max_{i=1,n}\overline{x_{1i}}}{\min_{i=1,n}\underline{x_{2i}}}
\end{equation}

Будем исследовать поведение $ R $ в области, заданной $ R_{out} $.

\section{Реализация}
\quad Из языка Python 3.12.2 были использованы следующие модули:
\begin{itemize}
  \item ``numpy'' --- генерация множества чисел;
  \item ``matplotlib.pyplot'' --- построение и отображение графиков;
  \item ``typing'', ``annotations'' --- строгая типизация с аннотациями;
  \item ``os'' --- взаимодействие с ОС.
\end{itemize}

\section{Результаты}
\quad Данные были взяты из файлов \textsl{+0\_5V/+0\_5V\_13.txt} и
\textsl{-0\_5V/-0\_5V\_13.txt}. С коррекцией при помощи вспомогательных данных из
файла \textsl{ZeroLine/ZeroLine\_13.txt}. Размер выборок: $ 1024 $.
Интервальная выборка на основе изначальных строится по формулам:

\begin{equation}
  x = [x_0 - \varepsilon, x_0 + \varepsilon], \varepsilon = 2^{-14}
\end{equation}
где $ x_0 $ - точечное значение.

Сначала посмотрим на исходные интервальные выборки $ X_1, X_2 $.
\plot{./images/signal 1.png}{Исходная интервальная выборка $ X_1 $}{p:x1}

\plot{./images/signal 2.png}{Исходная интервальная выборка $ X_2 $}{p:x2}

Также построим графики частоты пересечений подынтервалов для исходных
выборок.
\plot{./images/signal 1 mode hist.png}{Частота пересечений подынтервалов с интервалами выборки $ X_1 $}{p:modaX1}

\plot{./images/signal 2 mode hist.png}{Частота пересечений подынтервалов с интервалами выборки $ X_2 $}{p:modaX2}
\quad Проанализируем выборки.

Индекс Жаккара: $ JC(X_1) = -0.9909, JC(X_2) = -0.9911 $.

Частота моды: $ max\mu(X_1) = 29, max\mu(X_2) = 26 $.

Оптимальный корректирующий множитель: $ k_0(X_1) = 109.375, k_0(X_2) = 111.666 $.

Мультипликативная мера: $ T(X_1) = 1.1623e^{-6}, T(X_2) = 1.0001e^{-6} $.

Верхняя и нижняя границы $ \underline{R} = -1.0082, \overline{R} = -1.0065 $.

Найдем оптимальные $ R $
(для наглядности на графиках изображёны более широкии интервалы значений $ R $).
\plot{./images/sum signal JС.png}{Зависимость $ JC $ от $ R $}{p:jaccard}

Оптимальное значение $ R $ относительно $ JC $ равно $ R_{opt} = [-1.0069, -1.0065] $
при $ JС(X) = -0.9911 $.
\plot{./images/sum signal k.png}{Зависимость $ k_0 $ от $ R $}{p:k_0}

Оптимальное значение $ R $ относительно $ k_0 $ равно $ R_{opt} = [-1.0069, -1.0067] $
при $ k_0(X) = 111.666 $.
\plot{./images/sum signal maxmu.png}{Зависимость частоты моды от $ R $}{p:max_mu}

Оптимальное значение $ R $ относительно $ max\mu $ равно $ R_{opt} = [[-1.009, -1.0088], [-1.0078, -1.0076]] $
при $ max\mu(X) = 55 $.
\plot{./images/sum signal mult measure.png}{Зависимость $ T $ от $ R $}{p:t}

Оптимальное значение $ R $ относительно $ T $ равно $ R_{opt} = -1.0077 $
при $ T(X) = 1.0578e^{-6} $.

Итоговая оценка совместности: $ [-1.0077, [-1.0082, -1.0065]] $.

\section{Выводы}
\quad Из полученных результатов можно заметить следующее. Из теории следует, что после совмещения двух выборок описанным образом нельзя получить более совместную выборку, чем худшая из двух
изначальных. Это означает, что

\begin{equation}
  JС(X) \leq min(JC(X_1), JC(X_2))
\end{equation}

\begin{equation}
  k_0(X) \geq max(k_0(X_1), k_0(X_2))
\end{equation}

\begin{equation}
  max\mu(X) \leq max\mu(X_1) + max\mu(X_2)
\end{equation}

На практике во всех трех случаев удалось достичь равенства. Также можно заметить, что
данное правило не распространяется на мультипликативную меру. Более того, только для
мультипликативной меры $ R_{opt} $ задается точным числом, а не интервалом/мульти-интервалом.

\section{Литература}
\begin{itemize}
  \item \href{https://elib.spbstu.ru/dl/2/s20-76.pdf/info}{Баженов А.Н.
    <<Интервальный анализ. Основы теории и учебные примеры: учебное пособие>>};
  \item \href{https://elib.spbstu.ru/dl/5/tr/2021/tr21-169.pdf/info}{Баженов
      А.Н. <<Естественнонаучные и технические применения интервального анализа:
    учебное пособие>>};
  \item \href{https://github.com/AlexanderBazhenov/Students}{Баженов А.Н.
    Репозиторий ``Students'' на GitHub};
\end{itemize}

\section{Приложения}
\quad Исходники лабораторной работы выложены на
\href{https://github.com/MeShootIn/interval-analysis/tree/lab-1}{GitHub}.

\end{document}
